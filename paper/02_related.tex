\section{Related Work}
\label{sec:related}

\subsection{Optical Character Recognition}

OCR is the process of classifying and encoding characters within digital images. Due to its importance in numerous fields, OCR is a well-matured topic computer vision where various successful approaches have been developed. Neural Networks (NN) are one of the techniques among these approaches, and is the strategy that we employ in the VTex pipeline. Previous works demonstrate the success that NNs have found in OCR. A study by Sabourin \cite{SABOURIN1992843} shows that a multilayer perceptron network (MLP) is able to outperform other techniques in OCR such as dynamic contour warping classifiers. The model proposed by Wang \cite{Wang} attempts to improve on previous models such as seq2seq, with an architecture including a CNN encoder, and RNN decoder, and soft attention mechanism. Several other papers have performed different variations of NN's, with varying degrees of success\cite{Peng, Genthial, Wang02}. In VTex, we propose our own set of models to achieve improvements in recognition accuracy.


\subsection{Virtual Drawing}

With the progression of hand tracking technology, many researchers have expressed interests in hand tracking as a means of computer interaction as hand tracking is a natural way to express one's ideas. Studies show that visual cues can improve the remote work experience greatly\cite{SH, TEO}. Kim et al. has experimented with 3D sketching using hand tracking by having the user build up "air scaffolds" to constrain planes to draw on\cite{KIM}. Chen, et al. developed an application to input Chinese characters using a smart glasses’s camera\cite{CHEN}. Gulati et al. created a real-time drawing application using only a single RGB webcam, MediaPipe Hands, and OpenCV\cite{MPDrawing}. 


